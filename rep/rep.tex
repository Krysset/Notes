\documentclass{article}
\usepackage{amsfonts}
\usepackage{MnSymbol}
\usepackage{graphicx}
\title{Repetition MVE200}
\author{Krysset}
\begin{document}
    \section{Bevis}
    Induktion på $n\geq 2$.\\
    Basfallet $n=2$ är Följdsats 1.\\
    \underline{Induktionssteg:}\\
    Om $p\mid a_{1}\ldots a_{n-1}, a_{n}$, så måste $p\mid a_{1} \ldots a_{n-1}$ eller $p\mid a_{n}$ (enligt Följdsats 1).\\
    Om $p\mid a_{n}$ är vi klara. Om $p\mid a_{1}\ldots a_{n-1}$ så finns ett $i\in \{1,\ldots , n_{-1}\}$ så att $p\mid ai$,
    enligt induktionsantagandet. Så oavsett finns ett $i\in \{1,\ldots ,n\}$ så att $p\mid a$. v.s.b
    \section{Senast}
    \underline{Sats} Om $a\mid bc$ och sgd(a,b)=1, så delar a c\\
    \underline{Följdsats 1} Om p är ett primtal och $p\mid ab$, så måste $p\mid a$ eller $p\mid b$\\
    \underline{Följdsats 2} \\
    \\
    \subsection{Aritmetikens fundamentalsats}
\end{document}
\section*{P1}
\paragraph{21}~\\
Lös olikheten $x^2-2x\leq 0$. Svara i termer av intervall.\\
\underline{Notera:} $f(x)=x^2-2x$ är kontinuerlig på hela reella linjen.
\paragraph{Lösning:}~\\
Först löser vi nollställen till vänsterledet, alltså $x^2-2x$.
Vi ställer upp följande:
\begin{equation*}
    x^2-2x=0\Leftrightarrow x(x-2)=0
\end{equation*}
Då är våra lösningar $x=0$ och $x=2$.
Med denna information gör vi en tabell.
\begin{equation*}
    \begin{matrix}
                   &   & 0 &   & 2 &   \\
        x          & - & 0 & + & + & + \\
        x-2        & - & - & - & 0 & + \\
        \text{Tot} & + & 0 & - & 0 & +
    \end{matrix}
\end{equation*}
Då $x$ ska vara mindre än $0$ letar vi efter intervallet i tabellen som uppfyller det kravet.
I denna uppgiften var det $[0,2]$
\underline{Svar:} $[0,2]$
\section*{P2}
\paragraph{29}~\\
Finn intercept och lutning till linjen $\sqrt{2}x-\sqrt{3}y=2$. Skissa grafen till linjen.
\paragraph{Lösning}~\\
Linjen korsar x och y axeln då y respektive x är 0.
Alltså:
\begin{itemize}
    \item $x=0$ ger: $0-\sqrt{3}y=2\Leftrightarrow y=-\frac{2}{\sqrt{3}}$
    \item $y=0$ ger: $\sqrt{2}x-0=2\Leftrightarrow x=\frac{2}{\sqrt{2}}=\sqrt{2}$
\end{itemize}
Nu har vi två punkter där linjen korsar x- och y-axeln, $(0, -\frac{2}{\sqrt{3}})$ och $(\sqrt{2}, 0)$.
Med $\frac{\Delta y}{\Delta   x}$ kan vi räkna ut lutningen.
\begin{equation*}
    \frac{-(-\frac{2}{\sqrt{3}})+0}{0-\sqrt{2}}=
    \frac{\frac{2}{\sqrt{3}}}{\sqrt{2}}=
    \frac{\sqrt{2}}{\sqrt{3}}
\end{equation*}
\underline{Svar:} $\frac{\sqrt{2}}{\sqrt{3}}$

\paragraph{33}~\\
Finn skärningspunkterna till linjerna $3x+4y=-6$ och $2x-3y=13$.
\paragraph{Lösning:}~\\
Vi ställer upp de båda linjerna i ett ekvationssystem.
Man kan lösa detta genom gausselimination men vi gör det genom substitution.
\begin{equation*}
    \begin{matrix}
        3x+4y=-6 \\
        2x-3y=13
    \end{matrix}
\end{equation*}
Vi separerar $x$:
$3x+4y=-6\Leftrightarrow 3x=-6-4y\Leftrightarrow x=-2-\frac{4}{3}y$\\
Vi substituerar $x$ med det vi fick:
\begin{equation*}
    2x-3y=13\Leftrightarrow
    2\cdot(2-\frac{4}{3}y)-3y=13\Leftrightarrow
    -4-\frac{8}{3}y-\frac{9}{3}=13\Leftrightarrow
    -\frac{17}{3}y=17\Leftrightarrow
    -\frac{1}{3}y=1\Leftrightarrow y=-3
\end{equation*}
Då vi vet vad $y$ är kan vi räkna ut punktens x värde med ytterliggare en substitution.
$4x+4(-3)=-6\Leftrightarrow 3x=6\Leftrightarrow x=2$\\
\underline{Svar:} $(2, -3)$

\section*{A1}
\paragraph{13}~\\
Bestäm absolutbeloppet och argument för $z=\sqrt{3}-i$

\paragraph{Lösning}~\\
Absolutbeloppet av ett komplext tal räknas ut genom att betrakta talet
som en punkt i det komplexa talplanet och räkna ut avståndet från origo
till punkten.\\
Alltså använder vi pythagoras sats:
$|z|^2=(\sqrt{3})^2+1^2=3+1=4\Leftrightarrow |z|=\sqrt{4}=2$.\\
Nu ska vi räkna ut $arg(z)$.
Vi får en triangel som har sidorna $\sqrt{3},1,2$.
Vinkeln kan man då räkna ut med $tan(\frac{1}{\sqrt{3}})=30^\circ$.
\underline{Svar:} $|z|=2$ och $arg(z)=30^{circ}$

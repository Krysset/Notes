\chapter{Mängder och delmängder}
Mängder och delmängder är ett fundamentallt område inom matematik, alltså är det väldigt viktigt att kunna detta!

En mängd är en samling väldefinierade objekt.
Dessa objekt brukar kallas för \underline{element}.

En mängd $A$ bestående av elementen $a_1$, $a_2$, ... , $a_n$ skrivs som $A=[a_1,a_2, ... ,a_3]$
Om $A$ och $B$ är två olika mängder så betecknar $A\cup B$ alla element som tillhör $A$ eller $B$.
$A\cap B$ alla element som tillhör $A$ \underline{och} $B$.
Konstruktionen $A\cup B$ kallas för \underline{unionen} av $A$ och $B$ och $A\cap B$ kallas för \underline{snittet}.

Ett vanligt sätt att visualisera mängder är att genom så kallade \underline{venndiagram}:\\
\includegraphics[scale=0.5]{lessons/lesson01/imgs/img01.png}
Ett par saker till
\begin{itemize}
    \item $\emptyset=\{\}$, den tomma mängden
    \item $A^c$ alla element som inte finns i $A$ (kallas \underline{komplementet})
\end{itemize}

\paragraph{Talmängder:} Mängder vars element är tal\\

Några viktiga talmängder som är grundläggande i matetmatik är:
\begin{itemize}
    \item $\mathbb{N}=\{0,1,2,3,...\}$ de \underline{naturliga talen}
    \item $\mathbb{Z}=\{...,-2-,-1,0,1,2,...\}$ \underline{heltalen}
    \item $\mathbb{Q}=\{\text{Alla talen på formen }\frac{p}{q}\}$, där $p,q\in\mathbb{Z},q\neq 0$
    \item $\mathbb{R}=\{\text{Alla decimaltal}\}$ de \underline{reella talen}
    \item $\mathbb{C}=\{\text{alla tal }a+ib\}$, de \underline{komplexa talen}
\end{itemize}

Inom matematisk analys är mängderna $\mathbb{R}$ och $\mathbb{C}$ speciellt i fokus.

\chapter{Intervall}
\paragraph{}
Ett \underline{intervall} är en delmängd av $\mathbb{R}$ som innehåller
minst två tal och \underline{alla} tal mellan två av sina element.\\
Mer konkret:\\
Öppet intervall
%\begin{wrapfigure}{l}{0.4\textwidth}
%bilder på respektive intervall
%infoga bild 2-4
, dvs $\{x\in\mathbb{R}:a<x<b\}$ skrivs $(a,b)$\\
, dvs $\{x\in\mathbb{R}:a\leq x\leq b\}$ skrivs $[a,]$\\
, dvs $\{x\in\mathbb{R}:a\leq x\leq b\}$ skrivs $[a,)$\\
%\end{wrapfigure}

\paragraph{Ex:}
Lös olikheten $\frac{x}{2}\geq 1+ \frac{4}{x}$ och uttryck svaret som ett intervall eller en union av flera intervall.
\paragraph{Lösning:}
Måste försöka skriva om olikheten till faktoriserad form!
\begin{equation*}
    \frac{x}{2}\geq 1+\frac{4}{x}\Leftrightarrow
    \frac{4+x}{x}\Leftrightarrow
    \frac{x}{2}-\frac{4+x}{x}\geq 0\Leftrightarrow
    \frac{x^2-2x-8}{2x}\geq 0
\end{equation*}
Hitta nollställena till $x^2-2x-8$ genom kvadratkomplettering!
\begin{equation*}
    x^2-2x-8=0 \Leftrightarrow
    x^2-2\cdot 1\cdot x+1-1-8=0 \Leftrightarrow
    (x-1)^2-9=0\Leftrightarrow
    x=1\pm \sqrt{9}=1\pm 3\Leftrightarrow
    x=4\text{ eller } x=-2
\end{equation*}
Kan nu skriva om $\frac{x^2-2x-8}{2x}\geq 0$ som $\frac{(x-4)(x+2)}{2x}\geq 0$.
Härifrån kan man använda metoden med teckenstudium:
$\begin{matrix}
                     &   & -2 &   & 0 &   & 4 &   \\
        \frac{1}{2}x & - & -  & - & | & + & + & + \\
        x-4          & - & -  & - & - & - & - & + \\
        x+2          & 0 & +  & + & + & + & + & + \\
        \text{Tot}   & - & 0  & + & | & - & 0 & +
    \end{matrix}$
Ser att $\frac{x^2-2x-8}{2x}\geq 0$ uppfylls i intervallen $[-2,0]$ och $[4,\infty)$ och kan skriva lösningen som $[-2,0)\cup[4,\infty)$.
\paragraph{\underline{Absolutbelopp}}
Absolutbelopp av ett tal $x\in\mathbb{R}$ definieras som:
\begin{equation*}
    %ska vara l-bracket
    |x|=\begin{matrix}
        x \text{, om } x\geq 0 \\
        -x \text{, om } x\leq 0
    \end{matrix}
\end{equation*}
Följande tolkning gäller:
Givet ett tal $a\in\mathbb{R}$ så gäller för alla $x\in\mathbb{R}$ att $|x-a|=$ avståndet mellan $x$ och $a$.
Vidare gäller också, givet ett fixt tal $D\geq 0$, att
$|x-a|\begin{matrix}< \\=\\>\end{matrix}D\Leftrightarrow$
mängden av alla $x\in\mathbb{R}$ vars avst. till $a$ är $\begin{matrix}<\\=\\>\end{matrix}D$,
dvs $|x-a|\begin{matrix}
        < \\=\\>
    \end{matrix}D\Leftrightarrow\begin{matrix}
        a-D<x<a+D \\
        x=a-D     \\
        x<a-D,x>a+D
    \end{matrix}$
\paragraph{Ex:} (P1.41)\\
Lös olikheten $|x+1| >|x-3|$ genom att tolka avs som ett avst. på talaxeln.
\paragraph{Lösning}
$|x+1| = |x-(-1)|=$ "avst mellan $x$ och $(-1)$"\\
$|x-3|=$ "avst. mellan $x$ och $3$"\\
Så "avst. mellan $x$ och $(-1)$" $ > $ "avst. mellan $x$ och $3$"
%infoga bild 5
Till höger om $1$ så kommer $x$ \underline{alltid} att vara längre från $(-1)$ än $3$.

\chapter{Komplexa tal}
Ett komplext tal $z\in\mathbb{C}$ kan alltid skrivas på formen $z=a+i\cdot b$ där\\
\begin{itemize}
    \item $a$ kallas för \underline{realdelen} av $z$ $Re(z)$
    \item $b$ kallas för \underline{imaginärdelen }av $z$ $Im(z)$
\end{itemize}

Den imaginära enheten $i$ löser definitionsmässigt ekv. $x^2+1=0$, dvs $i=\sqrt{-1}$
Rent visuellt kan man betrakta ett komplext tal $a + ib$ som en punkt i det komplexa talplanet.
%infoga bild 6
Det gäller att $r^2= |a+ib|^2= \vdots =a^2+b^2$.
Givet $r$ och argumentet $\theta$ kan \underline{alla} kompexa tal skrivas $z=r(cos(\theta)+i\cdot sin(\theta))$

\chapter{Linjär approximation och Taylorutveckling}
Det är vanligt att funktioner modellerar de globala egenskaperna av olika system när teori för tillämpningar utvecklas.
Detta leder ofta till hög komplexitet.
Ibland är man dock bara intresserad av hur systemet beter sig av en viss punkt, säg $x=a$.
Vi kan då använda oss av linjär approximation för att underlätta beräkningar.
Vi vet att den linjära approximationen av $f$ i $x=a$, alltså tangentlinjen, går igenom punkten $(a,f(a))$ och har lutning $f'(a)$.
Detta innebär att $\frac{y-f(a)}{x-a}=f'(a)\Leftrightarrow y=f(a)+f'(a)\cdot (x-a)$.
Alltså, givet en deriverbar funktion $f$ så gäller för $x$-värden "nära" $a$ att $f(x)\approx L(x)=f(a)+f'(a)\cdot (x-a)$.
Hur stort fel innebär denna approximation?
Vi behöver den generaliserande medelvärdessatsen för derivator för att besvara detta.
\begin{tpastel}
    Om $f$ och $g$ är kontinuerliga på $[a,b]$, deriverbar på $(a,b)$ samt att $g'(x)\neq 0$ i $(a,b)$ så finns ett tal så att $\frac{f(b)-f(a)}{g(b)-g(a)}=\frac{f'(c)}{g'(c)}$.
\end{tpastel}
Felet av att approximationen $f(b)$ genom linjen approx $L$ i $x=a$ ges av:
$$E=f(b)-L(b)=f(f)-f(a)-f'(a)\cdot (b-a)$$
Tänk på $E$ som en funktion av $b$ och notera att $E(a)=0$ (dvs $b=a$).
Använd den genom medelvärdessatsen för $E(b)$ och $(b-a)^2$ genom kvoten:
$$\frac{E(b)}{(b-a)^2}=\{E(a)=0\}=\frac{E(b)-E(a)}{(b-a)^2-(a-a)^2} \Rightarrow E(b)=\frac{f''(s)}{2}\cdot (b-a)^2$$
Alltså, felet i att approximera $f(x)$ med $L(x)$ kan beräknas som $E(x)=\frac{f''(x)}{2}(x-a)^2$ för något tal $s$ mellan $a$ och $x$ ($a<s<x$ eller $x<s<a$).
\begin{epastel}
    \textbf{4.9.12}: En kub har sidlängden 20cm.
    Ungefär hur mycket måste denna minska om volymen ska minska med 12 cm$^3$?\\
    \textbf{Lösning:} Vi vet att $x=20$ cm och vill att $\Delta V=-12$ cm$^3$.
    Vi använder linjär approximation av $V$ kring $x=20$ cm.
    $L(x)=V(20)+V'(20)\cdot(x+20)$, $\Delta V = V(20)-L(x)=V'(20)\cdot(x-20)=V'(20)\cdot\Delta x$. (\dots)
    Kubens sidlängd ska alltså minska ed cirka $1/100=0.01$ cm.
\end{epastel}
Går det att approximera en funktion $f$ i omgivningen av en punkt $x=a$ på ett bättre sätt än linjärisering?
Vilka egenskaper hos linjäriseringen $L(x)$ gör den till en bra approximation?
$L(x)=f(a)+f'(a)\cdot(x-a)\Rightarrow L(a)=f(a)$ och $L'(a)=f'(a)$.
Kan man hitta en approximation där även andraderivatan stämmer? Ja!
$P(x)=f(a)+f'(a)\cdot(x-a)+\frac{f''(a)}{2}(x-a)^2\Rightarrow P(a)=f(a)$, $P(a)=f'(a)$\\
$P''(a)=0+0+\frac{f''(a)}{2}\cdot 2=f''(a)$
På liknande sätt kan man fortsätta att bygga på med högre ordningens termer.
Det gäller att $\frac{d^n}{dx^n}[(x-a)^n]=\{$kedjeregeln$\}=n\cdot\frac{d^{n-1}}{dx^{n-1}}[(x-a)^{n-a}]=\dots=3\cdot 2\cdot 1$.
Talet $n\cdot (n-1)\cdot\dots\cdot 2\cdot 1$ för $n\in\mathbb{N}$ kallas för $n$-fakultet och skrivs enklare som $n!$.
Man definierar $0!=1$.
Detta gör att man enkelt hittar en approximation $P_n$ till $f$ nära vilken punkt $x=a$ som helst (givet att $f$ är $n$ gånger deriverbar där)
där alla derivator upp till ordning $n$ överensstämmer som $P_n(x)=f(a)+f'(a)\cdot(x-a)+\frac{f''(a)}{2}(x-a)^2+\dots +\frac{f^{(n)}}{n!}(x-a)^n$.
(=$\sum^n_{i=0}\frac{f^{(i)}(a)}{i!}\cdot(x-a)^i$).
Approximation $P_n$ kallas för Taylorpolynomet av grad $n$ runt punkten $x=a$ (eller $n$:te gradens Taylorutveckling).
I specialfallet då $a=0$ kallar man ibland Taylorutvecklingen för Maclaurinutveckling.
Högre ordningens Taylorutvecklingar innebär bättre ock bättre approximationer av $f$ runt $x=a$ som också funkar längre och längre från $x=a$.
Givet $n$:te gradens Taylorpolynom runt $x=a$, alltså $P(x)$, hur bra är approximationen $P(x)\approx f(x)$?
Vi kan visa att feltermer $E_n(x)=\frac{f^{(n+1)}(s)}{(n+1)!}\cdot (x-a)^{n+1}$ för något tal tal $s$ mellan $a$ och $x$.
$E_n(x)$ kallas för Lagranges restterm.
\begin{epastel}
    \textbf{4.10.10}: Använd andra ordningens Taylorpolynom $P_2(x)$ för att approximera $\sqrt{61}$.
    Uppskatta även storleken av felet och ge ett intervall där det sanna värdet ligger.\\
    %\textbf{Lösning:} Låt $f(x)=\sqrt{}$
\end{epastel}

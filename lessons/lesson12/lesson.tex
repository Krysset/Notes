Givet en funktion $f$ och en punkt $x=a$,
där $f$ är tillräckligt många gånger deriverbar, lärde vi oss att:
\begin{equation*}
    f(x)=p_n(x)+E_n(x)=f(a)+f^\prime(a)(x-a)+...+\frac{f^{(n)}(a)}{n!}(x-a)^n+\frac{f^{n+1}(s)}{(n+1)!}(x-a)^{n+1}
\end{equation*}
för något tal $s$ mellan $a$ och $x$.
Polynomet $P_n(x)$ approx. funktionen $f$ i närheten av $x=a$.
Ofta är man inte intresserad av det exakta uttrycket för felet $E_n(x)$,
utan bara "hur snabbt det växer" i takt med att $x$ rör sig från $a$.
Smidigt att använda så kallad O-notation (dvs. ordo-notation eller stora O-notation).\\
Man skriver att $f(x)=O(u(x))$ då $x\to a$ om det finns tal $K>0$ och $\delta>0$ sådana att $|f(x)|\leq K\cdot|u(x)|$ då $0< |x-a| <\delta$.
På liknande sätt, om $f(x)=g(x)+O(u(x))$ då $x\to a$ så betyder det att $f(x)-g(x)=O(u(x))$.
För Taylor-utv. gäller alltså att $f(x)=P_n(x)+O((x-a)^{n+1})$.\\
Några räkneregler för O:
\begin{itemize}
    \item $C\cdot O(u(x))=O(u(x)),\forall C>0$
    \item $O(f(x))\cdot O(g(x))=O(f(x)\cdot g(x))$
    \item $O(x^m)+O(x^n)=O(x^n)$ om $n\geq m$ och $x\to\infty$
    \item $O(x^m)+O(x^n)=O(x^m)$ om $n\geq m$ och $x\to 0$
    \item Om $f(x)=O((x-a)^k\cdot u(x))$ då $x\to a$ så är $\frac{f(x)}{(x-a)^k}=O(u(x))$ då $x\to a$.
\end{itemize}

\paragraph{Sats (Om taylor-polynom och O)}
Om $f(x)=Q_n(x+O((x-a)^{n+1}))$ då $x\to a$ och $Q_n$ är ett polynom av grad $n$ så är $Q_n=$Taylorpolynomet av $f$ runt $x=a$.
\\\\
Kan förstå l'hopitals regel bättre med hjälp av taylor-polynom och O-notation!
Vi har lärt oss att om gränsvärdet $\lim_{x\to a}\frac{f(x)}{g(x)}$ är av typen $\frac{0}{0}$ och $g^\prime\neq 0$ i en omgivning av $x=a$ så kan man istället föröska beräkna $\lim_{x\to a}\frac{f^\prime(x)}{g^\prime(x)}$.
Om det senare gränsvärdet konvergerar och \underline{inte} är av typen $\frac{0}{0}$ så konvergerar det förra mot samma sak.
\paragraph*{Varför?}~\\
Betrakta första ordningens Taylorutveckling av $f$ och $g$ runt $x=a$:
\begin{equation*}
    f(x)=f(a)+f^\prime(a)\cdot (x-a)+O((x-a)^2)
\end{equation*}
\begin{equation*}
    g(x)=g(a)+g^\prime(a)\cdot (x-a)+O((x-a)^2)
\end{equation*}
\begin{equation*}
    \Rightarrow\lim_{x\to a}\frac{f(x)}{g(x)}=\lim_{x\to a}\frac{f(a)+f^\prime(a)\cdot (x-a)+O((x-a)^2)}{g(a)+g^\prime(a)\cdot (x-a)+O((x-a)^2)}=
\end{equation*}
\begin{equation*}
    =\{f(a)=g(a)=0 \text{ enl. förutsättning}\}=\lim_{x\to a}\frac{f^\prime(a)\cdot (x-a)+O((x-a)^2)}{g^\prime(a)\cdot (x-a)+O((x-a)^2)}=
\end{equation*}
\begin{equation*}
    =\{\text{Bryt ut }(x-a)\text{ ur täljare och nämnare}\}=\lim_{x\to a}\frac{f^\prime(a)+O(x-a)}{g^\prime(a)+O(x-a)}=\frac{f^\prime(a)}{g^\prime(a)}
\end{equation*}
Och om l'Hopital inte funkar, dvs. $\lim_{x\to a}\frac{f^\prime(x)}{g^\prime(x)}$ är av typen $\frac{0}{0}$?
Texta då istället $\lim_{x\to a}\frac{f^{\prime\prime}(x)}{g^{\prime\prime}(x)}$.
Om det gränsvärdet inte är av typen $\frac{0}{0}$ så kommer $\lim_{x\to a}\frac{f(x)}{g(x)}=\lim_{x\to a}\frac{f^{\prime\prime}(x)}{g^{\prime\prime}(x)}$.
\paragraph*{Varför?}~\\
Taylorutveckling till andra ordningen runt $x=a$ ger:
\begin{equation*}
    f(x)=f(a)+f^\prime(a)(x-a)+\frac{f^{\prime\prime}(a)}{2}(x-a)^2+O((x-a)^3)
\end{equation*}
\begin{equation*}
    g(x)=g(a)+g^\prime(a)(x-a)+\frac{g^{\prime\prime}(a)}{2}(x-a)^2+O((x-a)^3)
\end{equation*}
Samma räkning som innan ger att $\lim_{x\to a}\frac{f(x)}{g(x)}=...=\frac{f^{\prime\prime}(a)}{g^{\prime\prime}(a)}$

\paragraph{Ex (kompendie övn. 9.5)} Beräkna gränsvärdet $\lim_{x\to 0}(\frac{1}{sin^2(x)}-\frac{1}{x^2})$
\subparagraph{Lösning} Gränsvärdet är icke-trivialt då direkt insättning av $x=0$ ger $\infty-\infty$.
Börja med lite omskrivningar.
\begin{equation*}
    \frac{1}{sin^2(x)}-\frac{1}{x^2}=\frac{x^2}{x^2sin^2(x)}-\frac{sin^2(x)}{x^2sin^2(x)}=\frac{x^2-sin^2(x)}{x^2sin^2(x)}=
\end{equation*}
\begin{equation*}
    \{cos(2x)=cos^2(x)-sin^2(x)=(1-sin^2(x))-sin^2(x)=1-2sin^2(x)
\end{equation*}
\begin{equation*}
    \text{ så } sin^2(x)=\frac{1-cos^2(x)}{2}\}=\frac{x^2-\frac{1-cos(2x)}{2}}{x^2\frac{1-cos(2x)}{2}}=\frac{2x^2-1+cos(2x)}{x^2-x^2cos(2x)}
\end{equation*}
Från standardutvecklingar vet vi att
\begin{equation*}
    cos(x)=1-\frac{x^2}{2!}+\frac{x^4}{4!}-\frac{x^6}{6!}+...+O(x^{2n+2})
\end{equation*}
\begin{equation*}
    \Rightarrow cos(2x)=1-\frac{(2x)^2}{2!}+\frac{(2x)^4}{4!}-...
\end{equation*}
Använd denna utveckling till ordning 4 i täljaren och ordning 2 i nämnaren!
\begin{equation*}
    \lim_{x\to 0}(\frac{1}{sin^2(x)}-\frac{1}{x^2})=\lim_{x\to 0}\frac{2x^2-1+cos(2x)}{x^2-x^2cos(2x)}=
\end{equation*}
\begin{equation*}
    \lim_{x\to 0}\frac{2x^2-1+(1-\frac{(2x)^2}{2!}+\frac{(2x)^4}{4!}+O(x^6))}{x^2-x^2(1-\frac{(2x)^2}{2!}+O(x^4)}=\lim_{x\to 0}\frac{2x^2-1+1-2x^2+\frac{2}{3}x^4+O(x^6)}{x^2-x^2+2x^4+O(x^6)}=
\end{equation*}
\begin{equation*}
    \lim_{x\to 0}\frac{\frac{2}{3}x^4+O(x^6)}{2x^4+O(x^6)}=\lim_{x\to 0}\frac{x^4}{x^4}\cdot\frac{\frac{2}{3}+O(x^2)}{2+O(x^2)}=\frac{\frac{2}{3}}{2}=\frac{1}{3} \text{ }\Box
\end{equation*}

\paragraph{Ex} Använd Maclaurinutveckling för $sin(x)$ för att beräkna Maclaurinpolynomet av ordning 4 till funktionen $arcsin(x)$.
\begin{equation*}
    sin(x)=x-\frac{x^3}{3!}+\frac{x^5}{5!}-\frac{x^7}{7!}+...+O(x^{2n+1})
\end{equation*}
\subparagraph{Lösning}
Eftersom att $arcsin$ är invers funktion till $sin$ (på intervallet $-\frac{\pi}{2}\leq x \leq \frac{\pi}{2}$) så gäller per definition att:
$arcsin(sin(x))=sin(arcsin(x))=x$ och $arcsin(-sin(x))=arcsin(sin(-x))=-x=\{arcsin(sin(x))=x\}=-arcsin(sin(x))$ så $arcsin(-sin(x))=arcsin(sin(x))=\{sin(x)=z\}$
dvs. $arcsin(-z)=arcsin(z)$.
Detta betyder att $arcsin$ är en udda funktion och utvecklingen vi söker måste därför vara på formeln $a_1x+a_3x^3+O(x^5)$ för några tal $a_1$ och $a_3$.
Vi vet att $sin(arcsin(x))=x$ eftersom (åter igen) $arcsin$ är invers till $sin$,
och vi vet att $sin(x)=x-\frac{x^3}{3!}+O(x^5)$ vilket leder till att
\begin{equation*}
    x=sin(arcsin(x))=(a_1x+a_3x^3+O(x^5))-\frac{(a_1x+a_3x^3+O(x^5))}{6}+O((a_1+a_3x^3+O(x^5))^5)
\end{equation*}
\begin{equation*}
    =(a_1x+a_3x^3+O(x^5))-(\frac{a_1^3}{6}x^3+O(x^5))+O(x^5)=a_1x+(a_3-\frac{a_1^3}{6})x^3+O(x^5)
\end{equation*}
Vilket bara kan vara sant om $a_1=1$ och $a_3-\frac{a_1^3}{6}=0\Rightarrow a_3=\frac{1}{6}$ och vi får att $\arcsin(x)=x+\frac{x^3}{6}+O(x^5)\Box$
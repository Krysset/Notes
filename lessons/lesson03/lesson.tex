\chapter{Talföljder och gränsvärden}
Studium av talföljder är ett av matematikens mest klassiska områden.
Vi har exempelvis:
\begin{itemize}
    \item Fibonacci-talföljden, $0,1,1,2,3,5,8,...$, återfinns i olika sammanhang i naturen.
    \item Primtalssekvensen, $2,3,5,7,11,13,17,...$, finns formel för att beskriva sekvensen? (olöst)
\end{itemize}
Ska försöka formalisera begreppen i synnerhet för oändligt långa talföljder.\\
Låt $\{a_1,a_2,a_3,...\}=\{a_n\}, n\in\mathbb{N}$ vara en godtycklig talföljd.
Man säger att $\{a_n\}$ är:
\begin{itemize}
    \item Begränsad ovan-/underifrån om det finns ett tal $L$ sådant att $a_n\leq L$/$a_n\geq L$ $\forall n=1,2,3,...$.
    \item Begränsad om den är begränsad \underline{både} ovan- och underifrån.
    \item Positiv/Negativ om $a_n\geq 0$/$a_n\leq 0, \forall n=1,2,...$
    \item Växande/Avtagande om $a_{n+1}\geq a_n$/$a_{n+1}\leq a_n,\forall n=1,2,3,...$
    \item Monoton om talföljden är antingen växande eller avtagande
    \item Alternerande om $a_{n+1}\cdot a_n <0,\forall n=1,2,3,...$
\end{itemize}
Ett viktigt begrepp för talföljder (och funktioner) är \underline{konvergens}, dvs. om talföljden "stannar av" och håller sig oförändrad om man bara kollar tillräckligt långt in i följden (dvs. n stort).
%infoga bild 1
Måste försöka precisera vad detta betyder ren matematiskt.
\paragraph{Definition} Konvergent talföld\\
Man säger att en talföljd ${a_n}$ konvergerar mot $L\in\mathbb{R}$ och skriver $\lim_{n \to \infty}a_n=L$, om det för varhe positivt tal $\varepsilon > 0 $ existerar ett positivt heltal $N$ så att det för alla $n\leq N$ gäller att $|a_n-L|\leq\varepsilon$.
\subparagraph{Intuitivt} $\{a_n\}$ konvergerar mot $L$ om \underline{alla} tal tillräckligt långt in i följden ligger godtyckligt nära talet $L$.
Av detta följder "enkelt" att:
\begin{itemize}
    \item om $\{a_n\}$ konvergerar så är den begränsad.
    \item om $\{a_n\}$ är begränsad ovanifrån och växande så är $\{a_n\}$ konvergent.
          Motsvarande för begränsad underifrån och avtagande.
\end{itemize}
Bra räknelagar:
\begin{itemize}
    \item $\lim_{n \to \infty}(\frac{a_n}{b_n})=\frac{\lim_{n \to \infty}a_n}{\lim_{n \to \infty}b_n}$, om $\lim_{n \to \infty}b_n\neq 0$
    \item om $a_n\leq b_n\leq c_n$ och $\lim_{n \to \infty}a_n=\lim_{n \to \infty}c_n=L\Rightarrow \lim_{n \to \infty}b_n=L$.
\end{itemize}

\paragraph{Ex (9.1.25)} Bestäm om möjligt det tal $L$ som $a_n=\sqrt{n^2+n}-\sqrt{n^2-1}$ konvergerar mot då $n\to\infty$
\subparagraph{Lösning} Det gäller att
\begin{equation*}
    \sqrt{n^2-n}-\sqrt{n^2-1}=\sqrt{(n+1)\cdot n}-\sqrt{(n+1)(n-1)}=\sqrt{n+1}\cdot(\sqrt{n}-\sqrt{n-1})=
\end{equation*}
\begin{equation*}
    \sqrt{n+1}\cdot\frac{(\sqrt{n}\sqrt{n-1})\cdot(\sqrt{n}+\sqrt{n-1})}{\sqrt{n}+\sqrt{n-1}}=\sqrt{n+1}\cdot\frac{(n-(n-1))}{\sqrt{n}+\sqrt{n-1}}=\frac{\sqrt{n+1}}{\sqrt{n}+\sqrt{n-1}}
\end{equation*}
\begin{equation*}
    \text{och } \frac{\sqrt{n+1}}{\sqrt{n}+\sqrt{n-1}}\leq\frac{\sqrt{n}}{\sqrt{n}+\sqrt{n}}=\frac{1}{2}
\end{equation*}
\begin{equation*}
    \frac{\sqrt{n+1}}{\sqrt{n}+\sqrt{n+1}}\leq\frac{\sqrt{n+1}}{\sqrt{n-1}+\sqrt{n-1}}\cdot\frac{\sqrt{n-1}}{\sqrt{n-1}}=\frac{\sqrt{n^2-1}}{2(n-1)}\leq\frac{\sqrt{n^2}}{2(n-1)}=
\end{equation*}
\begin{equation*}
    \frac{n}{2(n-1)}=\frac{1}{2(1-\frac{1}{n})}\overrightarrow{n\to\infty}\frac{1}{2}
\end{equation*}
så $\lim_{n\to\infty}\sqrt{n^2+n}-\sqrt{n^2-1}=\frac{1}{2}\Box$
\\
\paragraph{} Ett av de mest kraftfulla verktygen inom matematisk analys är \underline{gränsvärden} för funktioner, dvs $\lim_{x \to a}f(x),a\in\mathbb{R}$.
Det ger oss derivator, integraler, differentialekvationer, ...
Hur ska man definiera gränsvärdet $\lim_{x \to a}f(x)$?
Skulle kunna inspireras av definitionen för talföljder.
\paragraph{Definition (försök)} Man säger att $f(x)$ konvergerar mot värdet $L\in\mathbb{R}$ då $x$ går mot $a\in\mathbb{R}$ om det för varje talföljd $\{x_n\}$ s.a $\lim_{n\to\infty}x_n=a$ gäller att $\lim_{n\to\infty}f(x)=L$.
\\\\Bättre definition i liknande riktning är dock.
\paragraph{Definition} Man säger att $f(x)$ går mot gränsvärdet
$L\in\mathbb{R}$ då $x$ går mot $a\in\mathbb{R}$ och skriver
$\lim_{x\to a}f(x)=L$, om det för varje tal $\varepsilon > 0$
existerar ett annat tal $\delta > 0$ (som ev. beror av $\varepsilon$) s.a.
om $0< |x-a| <\delta$ så ligger $x$ i $f$s definitionsmängd och
$|f(x)-L|<varepsilon$.

\paragraph{Ex}
\begin{enumerate}
    \item $f\to L_1$, när $x\to a_1$?
          %infoga bild 2
          Ja! Går alltid att hitta $\delta>0$ s.a. $|f(x)-L_1|<\varepsilon$ oavsett $\varepsilon$.
    \item $f\to L_2$, när $x\to a_1$?
          %infoga bild 3
          Omöjligt att hitta $\delta>0$ s.a. $|f(x)-L_2|<\varepsilon$ om $\varepsilon$ litet.
\end{enumerate}

\paragraph{Ex (1.5.19)}Använd definitionen av gränsvärde för att bevisa att
\begin{equation*}
    \lim_{x\to 1}\sqrt{x}=1
\end{equation*}
\subparagraph{Lösning} Vill hitta $\delta > 0$ så att
$|\sqrt{x}-1|<\varepsilon$ så länge som $0< |x-1| <\delta$
(givet vilket $\varepsilon>0$ som helst).
Gäller att $|\sqrt{x}-1<\varepsilon\Leftrightarrow-\varepsilon<\sqrt{x}-1<\varepsilon\Rightarrow 1-\varepsilon<\sqrt{x}<1+\varepsilon$.
$\left\{\begin{matrix}
        \text{Om }0<\varepsilon<\sqrt{x}\leq 1: 1-\varepsilon<\sqrt{x}<1+\varepsilon\Rightarrow(1-\varepsilon)^2<x<(1+\varepsilon)^2 \\
        \text{Om }\varepsilon>1:1-\varepsilon<\sqrt{x}<1+\varepsilon\Rightarrow 0<x<(1+\varepsilon)^2
    \end{matrix}\right.$
Notera att $(1-\varepsilon)^2<x<(1+\varepsilon)^2$ \underline{alltid} implicerar att $1-\varepsilon<\sqrt{x}<1+\varepsilon$ dvs $|\sqrt{x}-1|<\varepsilon$.
\begin{equation*}
    (1-\varepsilon)^2<x<(1-\varepsilon)^2\Leftrightarrow 1-2\varepsilon+\varepsilon^3<x<1+2\varepsilon+\varepsilon^2\Leftrightarrow-\varepsilon(2-\varepsilon)<x-1<\varepsilon\cdot(2+\varepsilon)
\end{equation*}
%infoga bild 4
så \begin{equation*}
    -\varepsilon\cdot(2-\varepsilon)<x-1<\varepsilon\cdot(2-\varepsilon)\Rightarrow-\varepsilon(2-\varepsilon)<x-1<\varepsilon(2+\varepsilon)
\end{equation*}
om $\varepsilon<2$.
Välj därför $\delta=\varepsilon\cdot(2-\varepsilon)$ om $\varepsilon<2$.
För $\varepsilon\leq 2$, välj t.ex $\delta=1$ eftersom $|x-1|<1\Rightarrow-1<\sqrt{x}-1<0\Rightarrow-2<\sqrt{x}-1<2\Rightarrow|\sqrt{x}-1|<2\leq\varepsilon\Box$

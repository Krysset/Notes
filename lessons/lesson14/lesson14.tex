\paragraph{Proposition 8.6} Om $\bm{v}$ är en egenvektor med egenvärde $\lambda$ till $A$ då är även $c\bm{v},c\neq 0$, en egenvektor med egenvärde $\lambda$.
\subparagraph{Bevis} $A(c\bm{v})=c(A\bm{v})=c\lambda\bm{v}=\lambda(c\bm{v})\text{ }\blacksquare$

\paragraph{Sats 8.8} Om $\bm{v}$ är en egenvektor med egenvärde $\lambda$ för matrisen $A$ då är 
$\bm{v}$ en egenvektor med egenvärde $\lambda^m$ till matrisen $A^m$, $m\in\mathbb{Z}_+$.
Om $A$ är inverterbar då gäller $A^m\bm{v}=\lambda^m\bm{v}$ även för negativa heltal.
Speciellt gäller att $A^{-1}\bm{v}=\frac{1}{\lambda}\bm{v}$.

\clearpage
\section{Beräkning av egenvärden och vektorer}
Om $A\bm{v}=\lambda\bm{v}$ så får vi $A\bm{v}-\lambda\bm{v}=\bm{0}$ eller $(A-\lambda I_n)\bm{v}=\bm{0}$.
Men eftersom $\bm{v}\neq\bm{0}$ så betyder detta att ekvationen $(A-\lambda I_n)\bm{v}=\bm{0}$ har icke-triviala lösningar.\\
Det finns icke-triviala lösningar $\Leftrightarrow det(A-\lambda I_n)=0$.\\
$\lambda$ så att $det(A-\lambda I_n)=0$ är ett egenvärde!
Uttrycket $P(\lambda)=det(A-\lambda I_n)$ är ett polynom kallas det \underline{karakteristiska polynomet för $A$}.
$P(\lambda)$ har grad $n$ om $A$ är en $n\times n$-matris.
Om $\lambda$ är ett egenvärde då ges egenvektorerna till $\lambda$ av icke-triviala lösningar till $(A-\lambda I_n)\bm{v}=\bm{0}$.

\paragraph{Ex} Låt $A=\begin{pmatrix}-1&3\\0&2\end{pmatrix}$.
Hitta egenvärdena till $A$.
\subparagraph{Lösning} Vi tittar på 
\begin{equation*}
    A-\lambda I_2=
    \begin{pmatrix}-1&3\\0&2\end{pmatrix}-\begin{pmatrix}\lambda&0\\0&\lambda\end{pmatrix}=
    \begin{pmatrix}-1-\lambda&3\\0&2-\lambda\end{pmatrix}
\end{equation*}
\begin{equation*}
    P(x)=det(A-\lambda I_2)=(-1-\lambda)(2-\lambda)=(\lambda+1)(\lambda-2)
\end{equation*}
\begin{equation*}
    P(\lambda)=0\Leftrightarrow\begin{cases}
        \lambda=-1
        \lambda=2
    \end{cases}
\end{equation*}
Låt oss hitta egenvektorer till $\lambda=2$.\\
Vi löser $(A-\lambda I_2)\bm{v}=\bm{0}\Leftrightarrow
\begin{pmatrix}
    -3&3&0\\0&0&0
\end{pmatrix}
\thicksim
\begin{pmatrix}
    1&-1&0\\0&0&0
\end{pmatrix}$\\
$\bm{v}=\begin{pmatrix}x\\y\end{pmatrix}$.
Låt $y=t$.
Då är $x=t$ så $\bm{v}=\begin{pmatrix}t\\t\end{pmatrix}=t\begin{pmatrix}1\\1\end{pmatrix}$.

\paragraph{Ex} Hitta egenvärden till $A=\begin{pmatrix}1&1&2\\1&0&9\\2&9&-3\end{pmatrix}$.
\subparagraph{Lösning} 
\begin{equation*}
    P(\lambda)=det(A-\lambda I_3)=
det\begin{pmatrix}
    1-\lambda&1&2\\
    1&-\lambda&9\\
    2&9&-3-\lambda
\end{pmatrix}=
\end{equation*}
\begin{equation*}
    (1-\lambda)\begin{vmatrix}
        -\lambda&9\\9&-3-\lambda
    \end{vmatrix}-
    1\begin{vmatrix}
        1&9\\2&-3-\lambda
    \end{vmatrix}+
    2\begin{vmatrix}
        1&-\lambda\\2&9
    \end{vmatrix}=
\end{equation*}
\begin{equation*}
    (1-\lambda)(-\lambda(-3-\lambda)-81)-1(-3-\lambda-18)+2(9+2\lambda)=
\end{equation*}
\begin{equation*}
    (1-\lambda)(\lambda^2-3\lambda-81)+\lambda+21+18+4\lambda=
\end{equation*}
\begin{equation*}
    \lambda^2+3\lambda-81-\lambda^3-3\lambda^2+81\lambda+5\lambda+39=-\lambda^3-2\lambda^2+89\lambda-42
\end{equation*}
För att hitta egenvärden måste vi lösa $-\lambda^3-2\lambda^2+89\lambda-42=0$.
Detta är för krångligt med penna och papper...

\clearpage
\section{Spektralsatser (Avs 8.3)}
\paragraph{Sats 8.14} Egenvektorer till olika egenvärden är linjärt oberoende.

\paragraph{Följdsats 8.15} En kvadratisk matris med $k$ olika egenvärden har åtminstone $k$ linjärt oberoende egenvektorer.
Speciellt så har en $n\times n$-matris med $n$ olika egenvärden också en bas av egenvektorer.

\paragraph{Sats 8.16-8.17}
\begin{enumerate}[label=(\alph*)]
    \item En symmetrisk matris har enbart reella lösningar till $P(\lambda)=0$ där $P(\lambda)$ är det karakteristiska polynomet.
    \item Egenvektorer till olika egenvärden, för en symmetrisk matris, är ortogonala.
\end{enumerate}

\paragraph{Sats 8.18} (Spectralsatsen för symmetriska matriser)\\
Låt $A$ vara en $n\times n$-matris.
Då gäller att
\begin{equation*}
    A\text{ har }n\text{ ortogonala egenvektorer }\Leftrightarrow A\text{ är symmetrisk}
\end{equation*}

\section{Diagonalisering (Avs 8.4)}

\paragraph{Sats 8.20} Låt $A$ vara en $n\times n$-matris.
Det finns en diagonal matris $D$ och en inverterbar matris $P$ så att
\begin{equation*}
    A=PDP^{-1} \Leftrightarrow \text{ } A \text{ har } n \text{ linjärt oberoende egenvektorer}
\end{equation*}
Om det första påståendet gäller sägs $A$ vara \underline{diagonaliserbar}.
För att hitta $D$ och $P$ gör vi som följande:
\begin{enumerate}
    \item[] Hitta egenvärden $\lambda_1,\lambda_2,\ldots,\lambda_n$ med egenvektorer $\bm{v}_1,\bm{v}_2,\ldots,\bm{v}_n$
    \item[] Sedan låter vi $D=\begin{pmatrix}
        \lambda_1&&&0\\
        &\lambda_2&&\\
        &&\ddots&\\
        0&&&\lambda_n
    \end{pmatrix}
    P=\begin{pmatrix}
        \bm{v}_1&\bm{v}_2&\ldots&\bm{v}_n
    \end{pmatrix}$
\end{enumerate}

\paragraph{Sats 8.22} Låt $A$ vara en $n\times n $-matris.
Det finns en diagonal matris $D$  och en matris $P$ så att
\begin{equation*}
    A=PDP^{-1}\leftrightarrow A\text{är symmetrisk}
\end{equation*}

\paragraph{Ex} Beräkna egenvärden för $A=\begin{pmatrix}2&7&1\\0&3&6\\0&0&5\end{pmatrix}$
\subparagraph{Lösning} 
\begin{equation*}
    P(\lambda)=det(A-\lambda I_3)=det\begin{pmatrix}
        2-\lambda&7&1\\
        0&3-\lambda&6\\
        0&0&5-\lambda
    \end{pmatrix}
    =(2-\lambda)(3-\lambda)(5-\lambda)
\end{equation*}
Egenvärdena till $A$ är $\lambda=2,\lambda=3,\lambda=5$.
Egenvärden till:
\begin{enumerate}
    \item[$\lambda=2$:]~\\
        $\begin{pmatrix}
            0&7&1&0\\
            0&1&6&0\\
            0&0&3&0
        \end{pmatrix}
        \begin{cases}
            z=0\\
            y=0\\
            x=t
        \end{cases}
        \text{ egenvektor }
        \begin{pmatrix}
            1\\0\\0
        \end{pmatrix}$
    \item[$\lambda=3$:]~\\
        $\begin{pmatrix}
            -1&7&1&0\\
            0&0&6&0\\
            0&0&2&0
        \end{pmatrix}
        \begin{cases}
            z=0\\
            y=0\\
            -x+7t=0
        \end{cases}
        \text{ egenvektor }
        \begin{pmatrix}
            7\\1\\0
        \end{pmatrix}$
    \item[$\lambda=5$] ~\\
        $\begin{pmatrix}
            -3&7&1&0\\
            0&2&6&0\\
            0&0&0&0
        \end{pmatrix}
        \begin{cases}
            z=t\\
            -2y+6z=0\Leftrightarrow y=3t\\
            -3x+7y+z=0\Leftrightarrow x=\frac{22}{3}t
        \end{cases}
        \text{ egenvektor }
        \begin{pmatrix}
            \frac{22}{3}\\3\\1
        \end{pmatrix}=
        \begin{pmatrix}
            22\\
            9\\
            3
        \end{pmatrix}$
\end{enumerate}
En bas egenvektorer till $A$: $\begin{pmatrix}
    1\\0\\0
\end{pmatrix},\begin{pmatrix}
    7\\1\\0
\end{pmatrix},\begin{pmatrix}
    22\\
    9\\
    3
\end{pmatrix}$\\
$D=\begin{pmatrix}
    2&0&0\\
    0&3&0\\
    0&0&5
\end{pmatrix},
P=\begin{pmatrix}
    1&7&22\\
    0&1&9\\
    0&0&3
\end{pmatrix}$\\
Då gäller att $A=PDP^{-1}$

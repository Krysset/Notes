\chapter*{Separabla differentialekvationer}
Matematisk modellering handlar ofta om att använda generella principer\\
(fysikaliska, tekniska, hypotetiska) för att beskriva hur olika typer av processer fungerar.
Man vet sällan funktionen $f(x)$, man känner bara till de lagar och principer som den måste respektera.

\paragraph*{Ex} Vilken kastbana följer en toppad tennisboll (dvs. en boll som roterar framåt i luften)?
\subparagraph{Lösning}~\\
%Infoga bild 1
Då bollen färdas i luften utsätts den för tre olika krafter:
\begin{enumerate}
    \item gravitation $\bm{F}_G$
    \item luftmotstånd $\bm{F}_D$
    \item magnuskraften $\bm{F}_M$
\end{enumerate}
%Infoga bild 2
Det gäller att $|\bm{F}_G|=mg$, $|\bm{F}_D|=C_1\cdot\frac{|\bm{V}|^2}{2}$ och $|\bm{F}_M=C_2\cdot\frac{|\bm{V}|^2}{2+\frac{|\bm{V}|}{R\cdot w}}$ och av Newtons 2:a lag gäller:
\begin{equation*}
    m\cdot a=\bm{F}_G+\bm{F}_D+\bm{F}_M
\end{equation*}
Eftersom $v_x=\frac{dx}{dt}$, $v_y=\frac{dy}{dt}$, $a_x=\frac{d^2x}{dt^2}$, $a_y=\frac{d^2y}{dt^2}$ kan man skriva detta som två separata (kopplade) samband i $x$- och $y$-led:
\begin{equation*}
    (*)\left\lbrace
    \begin{matrix}
        \frac{d^2x}{dt^2}=-k\cdot v(C\cdot\frac{dx}{dt}+\frac{\frac{dy}{dt}}{2+\frac{v}{Rw}}) \\
        \frac{d^2y}{dt^2}=-k\cdot v(\frac{\frac{dx}{dt}}{2+\frac{v}{Rw}}-C\cdot\frac{dy}{dx})-g
    \end{matrix}
    \right.
\end{equation*}
och där $v=\sqrt{(\frac{dx}{dt})^2+(\frac{dy}{dt})^2}, C,k\in\mathbb{R}$.
Bollbanan ges av de funktioner $x(t)$ och $y(t)$ som uppfyller $(*)$.
\section{Separabla differentialekvationer}
En differentialekvation som kan skrivas på formeln: $\frac{dy}{dx}=f(x)\cdot g(y)$ kallas för \underline{separabel}.
Målet är att hitta ett uttryck för $y=y(x)$.
Lösningsmetodiken går ut på att separera $x$ och $y$ på var sin sida om likhetstecknet.
\begin{equation*}
    \frac{dy}{dx}=f(x)\cdot g(y)\begin{matrix}(?)\\\Leftrightarrow\end{matrix}
    dy=f(x)g(y)\, dx\Leftrightarrow
    \frac{1}{g(y)}\, dy=f(x)\, dx
\end{equation*}
Vilket betyder att
\begin{equation*}
    \int\frac{1}{g(y)}\, dy=\int f(x)\, dx
\end{equation*}
Kan man lösa dessa och sedan uttrycka $y$ som en funktion av $x$ så är man klar.

\paragraph{Ex (7.9.8)} Lös differentialekvationen $\frac{dy}{dx}=1+y^2$.
\subparagraph{Lösning}
Denna differentialekvation är separabel eftersom
\begin{equation*}
    \frac{dy}{dx}=1+y^2\Rightarrow
    \frac{1}{1+y^2}\, dy=1\, dx\Rightarrow
    \int\frac{1}{1+y^2}=\int\, dx=x+C,\, C\in\mathbb{R}
\end{equation*}
För den andra integralen har vi att:
\begin{equation*}
    \int\frac{1}{1+y^2}\, dy=
    \arctan(y)+D,\, D\in\mathbb{R}\Rightarrow
\end{equation*}
\begin{equation*}
    \arctan(y)+D=x+c\Leftrightarrow
    \arctan(y)=x+E\, (E=C-D)
\end{equation*}
och alltså gäller att
\begin{equation*}
    y=\tan(x+e)
\end{equation*}
Prova om det stämmer!
\begin{equation*}
    \frac{dy}{dx}=
    \frac{d}{dx}[tan(x+E)]=
    \frac{1}{\cos^2(x+E)}=
    \frac{\sin^2(x+E)+\cos^2(x+E)}{\cos^2(x+E)}=
\end{equation*}
\begin{equation*}
    1+\frac{\sin^2}{\cos^2(x+E)}=
    1+\tan(x+E)=
    \{y=\tan(x+E)\}=1+y^2\, \Box
\end{equation*}

\chapter{Första ordningens linjära differentialekvation}
"Första ordningen" syftar på att högsta ordningen derivata i differentialekvationen är $1$, dvs $y^\prime$.
En linjär sådan differentialekvation kan skrivas $y^\prime+p(x)\cdot y=q(x)$.
Hur lösa?\\
Vill hitta $y(x)$ så att $y^\prime+p(x)y=q(x)$ givet $p(x)$ och $q(x)$.
Antag att man kan beräkna den primitiva funktionen för $p(x)$, dvs. $\int p(x)\, dx$.
Då gäller att
\begin{equation*}
    \frac{d}{dx}[e^{\int p(x)\, dx}]=
    \{\text{kedjeregeln}\}=
    e^{\int p(x)\, dx}\cdot\frac{d}{dx}[\int p(x)\, dx]=
    e^{\int p(x)\, dx}\cdot p(x)+C
\end{equation*}
Om både vänster- och högerled i differentialekvationen multipliceras med $e^{\int p(x)\, dx}$ får vi
\begin{equation*}
    e^{\int p(x)\, dx}\cdot y^\prime+e^{\int p(x)\, dx}\cdot p(x)\cdot y=
    e^{\int p(x)\, dx}\cdot q(x)\Leftrightarrow
    e^{\int p(x)\, dx}\cdot y^\prime+\frac{d}{dx}[e^{\int p(x)\, dx}]\cdot y=
    e^{\int p(x)\, dx}\cdot q(x)
\end{equation*}
Produktregeln ger: $\frac{d}{dx}[e^{\int p(x)\, dx}\cdot y]=e^{\int p(x)\, dx}\cdot q(x)$.
Om man integrerar vänster- och högerled med avseende på $x$ fås:
\begin{equation*}
    e^{\int p(x)\, dx}\cdot y=\int e^{\int p(x)\, dx}\cdot q(x)\, dx
\end{equation*}
och man hittar lösningen $y(x)$ till differentialekvationen som
\begin{equation*}
    y(x)=e^{-\int p(x)\, dx}\cdot\int e^{\int p(x)\, dx}\cdot q(x)\, dx + C
\end{equation*}
Funktionen $e^{\int p(x)\, dx}$ i lösningsmetodiken kallas för den \underline{integrerande faktorn}.

\paragraph{Ex} Lös begynnelsevärdesproblemet
\begin{equation*}
    \left\lbrace
    \begin{matrix}
        y^\prime+\cos(x)\cdot y=2x\cdot e^{-\sin(x)} \\
        y(\pi)=0
    \end{matrix}
    \right.
\end{equation*}
\subparagraph{Lösning}
Använd metoden med integrerande faktor!
\begin{equation*}
    \int\cos(x)\, dx=\sin(x)
\end{equation*}
Den integrerande faktorn är alltså $e^{\sin(x)}$ och
\begin{equation*}
    e^{\sin(x)}\cdot y^\prime+e^{\sin(x)}\cdot \cos(x)\cdot y=
    e^{\sin(x)}\cdot 2x\cdot e^{-\sin(x)}=
    2x\Rightarrow
\end{equation*}
\begin{equation*}
    \frac{d}{dx}[e^{\sin(x)}\cdot y]=
    2x\Rightarrow
    \{\text{integrera}\}\Rightarrow
    e^{\sin(x)}\cdot y=
    \int 2x\, dx=
    x^2+C
\end{equation*}
och får att $y=e^{-\sin(x)}(x^2+C)$.
Vi vet också att $y(\pi)=0$ vilket används för att bestämma konstanten $C$.
\begin{equation*}
    y(\pi)=
    e^{-\sin(\pi)}(\pi^2+C)=
    e^0(\pi^2+C)=
    \pi^2+C\Rightarrow
    \pi^2+C=0\Leftrightarrow
    C=-\pi^2
\end{equation*}
och lösningen till problemet är
\begin{equation*}
    y(x)=e^{-\sin(x)}(x^2-\pi^2)\, \Box
\end{equation*}

\paragraph*{Ex (7.9.29)} Enligt Newtons andra lag kan hastigheten av en fritt fallande kropp i luft beskrivas som $m\cdot\frac{dv}{dt}=mg-hv^2$.
Talen $m$ och $g$ är kroppens massa respektive tyngdaccelerationen och termen $-h\cdot v^2$ modellerar luftmotståndet.
Givet att kroppen faller från vila i $t_0$ ($v(0)=0$).
Bestäm ett uttryck för $v(t)$.
Vad blir $\lim_{t\to\infty}v(t)$?
\subparagraph{Lösning}
$m\cdot\frac{dv}{dt}=mg-kv^2\Leftrightarrow g-\frac{k}{m}v^2$ är separabel!
\begin{equation*}
    \Rightarrow \frac{1}{g-\frac{k}{m}v^2}\, dv=dt\Rightarrow
    \int\frac{1}{g-\frac{k}{m}v^2}\, dv=
    \int\, dt=
    t+C,\, C\in\mathbb{R}
\end{equation*}
\begin{equation*}
    \int\frac{1}{g-\frac{k}{m}v^2}\, dv=
    \frac{m}{k}\int\frac{1}{\frac{mg}{k}-v^2}\, dv=
    \frac{m}{k}\int\frac{1}{(\sqrt{\frac{mg}{k}}-v)(\sqrt{\frac{mg}{k}}+v)}\, dv=
\end{equation*}
\begin{equation*}
    \{\alpha=\sqrt{\frac{mg}{k}}\}=
    \frac{m}{k}\int\frac{1}{(\alpha-v)(\alpha+v)}\, dv=
    \{\text{partialbråksuppdelning}\}=
\end{equation*}
\begin{equation*}
    \begin{vmatrix}
        \frac{1}{(\alpha-v)(\alpha+v)}=\frac{A}{\alpha-v}+\frac{B}{\alpha+v}\Rightarrow
        \begin{matrix}
            A=\frac{1}{2\alpha} \\
            B=\frac{1}{2\alpha}
        \end{matrix}
    \end{vmatrix}=
    \frac{m}{k}\cdot\frac{1}{2\alpha}\int\frac{1}{\alpha-v}+\frac{1}{\alpha+v}\, dv=
\end{equation*}
\begin{equation*}
    \frac{m}{k}\cdot\frac{1}{2\alpha}\int(\ln(|\alpha+v|)-\ln(|\alpha-v|))=
    \frac{m}{k}\cdot\frac{1}{2\alpha}\ln(|\frac{\alpha+v}{\alpha-v}|)\Rightarrow
    \frac{m}{k}\cdot\frac{1}{2\alpha}\ln(|\frac{\alpha+v}{\alpha-v})=t+C,\, C\in\mathbb{R}
\end{equation*}
\begin{equation*}
    V(0)=0\Rightarrow
    \frac{m}{k}\cdot\frac{1}{2\alpha}\ln(|\frac{\alpha+v}{\alpha-v})=
    0+C\Leftrightarrow
    C=0\, (\ln(1)=0)
\end{equation*}
så $\ln(|\frac{\alpha+v}{\alpha-v}|)=\frac{2\alpha tk}{m}=\{\alpha=\sqrt{\frac{mg}{k}}\}=2\sqrt{\frac{kg}{m}}\cdot t$.
För $v<\alpha$ så gäller att $\frac{\alpha+v}{\alpha-v}=e^{2\sqrt{\frac{kg}{m}}\cdot t}$ vilket efter lite räkningar ger $v(t)=\sqrt{\frac{mg}{k}\cdot\frac{e^{2\sqrt{\frac{kg}{m}}\cdot t}-1}{e^{2\sqrt{\frac{kg}{m}}\cdot t}+1}}$.
\begin{equation*}
    \frac{e^{2\sqrt{\frac{kg}{m}}\cdot t}-1}{e^{2\sqrt{\frac{kg}{m}}\cdot t}+1}\to 1\text{ då }t\to\infty
\end{equation*}
\begin{equation*}
    \text{så }\lim_{t\to\infty}v(t)=\sqrt{\frac{mg}{k}}\, \Box
\end{equation*}
\paragraph{Ex} Lös $\begin{cases}-x+z=3\\-2x-y+5z=-1\\2x+y=1\end{cases}$
\subparagraph{Lösning} Totalmatrisen ges av 
\begin{equation*}
    \begin{pmatrix}
        -1&0&1&3\\
        -2&-1&5&-1\\
        2&1&0&1
    \end{pmatrix} 
    \thicksim
    \begin{pmatrix}
        1&0&-1&-3\\
        0&-1&3&-7\\
        0&1&2&7
    \end{pmatrix}
    \thicksim
    \begin{pmatrix}
        1&0&-1&-3\\
        0&1&-3&7\\
        0&0&5&0
    \end{pmatrix}
\end{equation*}
\begin{equation*}
    \thicksim
    \begin{pmatrix}
        1&0&-1&-3\\
        0&1&-3&7\\
        0&0&1&0
    \end{pmatrix}
    \thicksim
    \begin{pmatrix}
        1&0&0&-3\\
        0&1&0&7\\
        0&0&1&0
    \end{pmatrix}
    \Leftrightarrow
    \begin{cases}
        x=3\\
        y=7\\
        z=0
    \end{cases}
\end{equation*}

\paragraph{Proposition 5.9} Om $T$ är totalmatrisen till ett linjärt ekvationssystem och 
$S$ fås från $T$ med en elementär radoperation, då har $S$ och $T$ samma lösningar.

\paragraph{Ex} Lös $\begin{cases}2x+y+3z=0\\x+y+z=6\\x+3y+2z=13\end{cases}$
\subparagraph{Lösning} Totalmatris:
\begin{equation*}
    \begin{pmatrix}
        2&1&3&0\\1&1&1&6\\1&3&2&13
    \end{pmatrix}
    \thicksim
    \begin{pmatrix}
        1&1&1&6\\
        0&-1&1&-12\\
        0&2&1&7
    \end{pmatrix}
    \thicksim
    \begin{pmatrix}
        1&1&1&6\\
        0&1&-1&12\\
        0&0&3&-17
    \end{pmatrix}
\end{equation*}
\begin{equation*}
    \Leftrightarrow
    \begin{cases}
        3z=-17\Leftrightarrow z=-\frac{17}{3}\\
        y-z=12\Leftrightarrow y = 12+z= 12\frac{17}{3}\\
        x+y+z=6\Leftrightarrow x=6-y-z=6-(12-\frac{17}{3})+\frac{17}{3}
    \end{cases}
    \Leftrightarrow
    \begin{cases}
        x=-7-\frac{2\cdot 17}{3}\\
        y=12-\frac{17}{3}\\
        z=-\frac{17}{3}
    \end{cases}
\end{equation*}

\paragraph{Definition} 
\begin{itemize}
    \item En matris sådan att varje rad har fler inledande nollor än raden ovan är på \underline{trappstegsform}
    \item Det första nollskilda elementet på en rad kallas för ett \underline{pivotelement}
    \item En kolumn som inte innehåller ett pivotelement kallas för en \underline{fri kolumn} (Inte den sista i kolumnen i en totalmatris)
\end{itemize}

\paragraph{Ex} Skriv 
$\begin{pmatrix}
    -4&-2&-9&-3&6\\
    4&2&12&8&2\\
    6&2&15&14&2
\end{pmatrix}$
på trappstegsform.
\subparagraph{Lösning}
\begin{equation*}\thicksim
    \begin{pmatrix}
        2&1&6&4&1\\
        -4&-2&-9&-3&6\\
        6&3&15&14&2
    \end{pmatrix}
    \thicksim
    \begin{pmatrix}
        2&1&6&4&1\\
        0&0&3&5&8\\
        0&0&-3&2&-1
    \end{pmatrix}
    \thicksim
    \begin{pmatrix}
        2&1&6&4&1\\
        0&0&3&5&8\\
        0&0&0&7&7
    \end{pmatrix}
\end{equation*}

\paragraph{Propostion 5.13} Varje matris kan reduceras till trappstegsform med hjälp av elementära radoperationer

\paragraph{Ex} Lös $\begin{cases}
    x+y+2z=3\\
    x+2y-z=1
\end{cases}$
\subparagraph{Lösning} Totalmatris:
\begin{equation*}
    \begin{pmatrix}
        1&1&2&3\\
        1&2&-1&1
    \end{pmatrix}
    \thicksim
    \begin{pmatrix}
        1&1&2&3\\
        0&1&-3&-2
    \end{pmatrix}
    \thicksim
    \begin{pmatrix}
        1&0&5&5\\
        0&1&-3&-2
    \end{pmatrix}
    \Leftrightarrow
    \begin{cases}
        x+5z=5\\
        y-3z=-2
    \end{cases}
\end{equation*}
Låt $z=t$, dvs. vilket reelt tal $t$ som helst, en parameter.\\
Då får vi 
$\begin{cases}
    x=5-5t\\
    y=-2+3t\\
    z=t
\end{cases}
\Leftrightarrow 
\begin{pmatrix}
    x\\y\\z
\end{pmatrix}
=
\begin{pmatrix}
    5\\-2\\0
\end{pmatrix}
+
t\begin{pmatrix}
    -5\\3\\1
\end{pmatrix}$

\paragraph{Proposition 5.16} Antag att ett ekvationssystem har reducerats till en totalmatris $T$ på trappstegform m.h.a elementära radoperationer.
För lösningarna gäller:
\begin{enumerate}
    \item Om det finns ett pivotelement i den sista kolumnen saknas lösningar
    \item Om det inte finns ett pivotelement i sista kolumnen då finns det lösningar och man får den genom att
        \begin{enumerate}
            \item sätta alla variabler som svarar mot fria kolumner till paramterar
            \item Börja i sista raden och gå uppåt och successivt lösa ut variablerna som svarar mot pivotkolumner.
        \end{enumerate}
        Det finns oändligt många lösningar om det finns fria kolumner. Annars finns en unik lösning.
\end{enumerate}

\paragraph{Ex} Lös $\begin{cases}
    x+y=2\\
    2x-y=4\\
    3x+2y=-3
\end{cases}$
\subparagraph{Lösning} 
\begin{equation*}
    \begin{pmatrix}
        1&1&2\\
        2&-1&4\\
        3&2&-3
    \end{pmatrix}
    \thicksim
    \begin{pmatrix}
        1&1&2\\
        0&-3&0\\
        0&-1&-9
    \end{pmatrix}
    \thicksim
    \begin{pmatrix}
        1&1&2\\
        0&1&9\\
        0&0&27
    \end{pmatrix}
\end{equation*}
Det finns pivotelement i sista kolumnen $\Rightarrow$ det saknas lösningar.

\paragraph{Ex} Lös $\begin{cases}
    2y+z=4\\
    x-z=1\\
    -2x+y=0
\end{cases}$
\subparagraph{Lösning}
\begin{equation*}
    \begin{pmatrix}
        0&2&1&4\\
        1&0&-1&1\\
        -2&1&0&0
    \end{pmatrix}
    \thicksim
    \begin{pmatrix}
        1&0&-1&1\\
        0&2&1&4\\
        0&1&-2&2
    \end{pmatrix}
    \thicksim
    \begin{pmatrix}
        1&0&-1&1\\
        0&1&-2&2\\
        0&0&5&0
    \end{pmatrix}
    \Leftrightarrow
    \begin{cases}
        x-z=1\Leftrightarrow x=1\\
        y-2z=2\leftrightarrow y=2\\
        z=0
    \end{cases}
\end{equation*}

\section{Kvadratiska system (Avs 5.4)}
\paragraph{Sats 5.20} Låt $T$ vara den totalmatrisen reducerad till trappstegsform för $A\bm{x}=\bm{b}$.
Följande är ekvivalent:
\begin{enumerate}
    \item Det finns en unik lösning till $A\bm{x}=\bm{b}$ för alla $\bm{b}$
    \item $T$ saknar fria kolumner
\end{enumerate}
Om 1 och 2 inte gäller då har $A\bm{x}=\bm{b}$ inga eller oändligt många lösnignar.

\paragraph{Sats 5.28} Om $A$ är en Kvadratisk matris så är föjande ekvivalent:
\begin{enumerate}
    \item $A\bm{x}=\bm{b}$ har en unik lösning för ala$\bm{b}$
    \item Man kan reducera $A$ till identitetsmatrisen m.h.a elementära radoperationer
    \item $A$ är inverterbar
\end{enumerate}
Om $A$ är inverterbar så är lösningen till $A\bm{x}=\bm{b}\Leftrightarrow \bm{x}=A^{-1}\bm{b}$\\
~\\
Om vi sätter $A$ och identitetsmatrisen $I_{n}$ i en stor matris $\begin{pmatrix}A&I_{n}\end{pmatrix}$ och 
reducerar den till $\begin{pmatrix}I_{n}&B\end{pmatrix}$ m.h.a elementära radoperationer, då är $B=A^{-1}$.\\
Om vi misslyckas är inte $A$ inverterbar.

\begin{figure*}
\paragraph{Ex} Hitta inversen till $A=\begin{pmatrix}1&2&-2\\2&1&1\\3&1&2\end{pmatrix}$.
\subparagraph{Lösning}~\\
$\begin{pmatrix}
    1&2&-2&1&0&0\\
    2&1&1&0&1&0\\
    3&1&2&0&0&1
\end{pmatrix}
\thicksim
\begin{pmatrix}
    1&2&-2&1&0&0\\
    0&-3&5&-2&1&0\\
    0&-5&8&-3&0&1
\end{pmatrix}
$\\

$\thicksim
\begin{pmatrix}
    1&2&-2&1&0&0\\
    0&1&-\frac{5}{3}&\frac{2}{3}&-\frac{1}{3}&0\\
    0&-\frac{15}{3}&\frac{24}{3}&-\frac{9}{3}&0&\frac{3}{3}
\end{pmatrix}
\thicksim
\begin{pmatrix}
    1&2&-2&1&0&0\\
    0&1&-\frac{5}{3}&\frac{2}{3}&-\frac{1}{3}&0\\
    0&0&\frac{24}{3}-\frac{15\cdot 5}{9}&-\frac{9\cdot 3}{3\cdot 3}+\frac{30}{9}&-\frac{15}{3}&\frac{3}{3}
\end{pmatrix}
$\\

$\thicksim
\begin{pmatrix}
    1&2&-2&1&0&0\\
    0&1&-\frac{5}{3}&\frac{2}{3}&-\frac{1}{3}&0\\
    0&0&-\frac{1}{3}&\frac{1}{3}&-\frac{15}{3}&\frac{3}{3}
\end{pmatrix}
\thicksim
\begin{pmatrix}
    1&2&-2&1&0&0\\
    0&1&-\frac{5}{3}&\frac{2}{3}&-\frac{1}{3}&0\\
    0&0&1&-1&15&-1  
\end{pmatrix}
$\\

$\thicksim
\begin{pmatrix}
    1&2&0&-1&30&-2\\
    0&1&0&\frac{2}{3}-\frac{5}{3}&-\frac{1}{3}+\frac{25\cdot 3}{3}&-\frac{5}{3}\\
    0&0&1&-1&15&-1
\end{pmatrix}
\thicksim
\begin{pmatrix}
    1&2&0&-1&30&-2\\
    0&1&0&-1&\frac{74}{3}&-\frac{5}{3}\\
    0&0&1&-1&15&-1
\end{pmatrix}
$\\
$
\thicksim
\begin{pmatrix}
    1&0&0&1&30-\frac{2\cdot 74}{3}&-2+\frac{10}{3}\\
    0&1&0&-1&\frac{74}{3}&-\frac{5}{3}\\
    0&0&1&-1&15&-1
\end{pmatrix}
\Rightarrow
A^{-1}=
\begin{pmatrix}
    -1&-\frac{58}{3}&\frac{4}{3}\\
    -1&\frac{74}{3}&-\frac{5}{3}\\
    -1&15&-1
\end{pmatrix}$
\end{figure*}
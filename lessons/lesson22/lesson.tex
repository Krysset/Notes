\chapter{Andra ordningens linjära differential ekvationer med konstanta koefficienter}
Ofta räcker första ordningens evkationer inte och vi behöver kunna hantera även högre ordningar.
Naturligt att studera andra ordningens linjära differentialekvationer som ett nästa steg,
dvs. något i stil med:
\begin{equation*}
    y^{\prime\prime}+p(x)\cdot y^\prime+q(x)\cdot y=f(x)
\end{equation*}
för givna funktioner $p(x)$, $q(x)$ och $f(x)$.
Detta problem är dock \underline{betydligt} svårare än dess motsvarighet av första ordningen och vi ska nöja oss med fallet där $p(x)$ och $q(x)$ är konstanter,
dvs.
\begin{equation*}
    (**)\, y^{\prime\prime}+p\cdot y^\prime+q\cdot y=f(x),\,p,q\in\mathbb{R}
\end{equation*}
\section*{Terminologi}
\underline{Om} funktionen $f(x)=0$ så säger man att differentialekvationen är \underline{homogen}.
I annat fall säger man att differentialekvationen är \underline{icke-homogen}.
Antag att $y_p$ är en lösning till en icke-homogen differentialekvation av typen $(**)$ och att $y$ är en lösning till mostvarande homogena ekvationen.
Då gäller att funktionen $y_p+y_h$ löser den icke-homogena differentialekvationen eftersom:
\begin{equation*}
    (y_p+y_h)^{\prime\prime}+p(y_p+y_h)^\prime+q(y_p+y_h)=
    y_p^{\prime\prime}+y_h^{\prime\prime}+p\cdot y_p^\prime+p\cdot y_h^\prime+q\cdot y_p+q\cdot y_h=
\end{equation*}
\begin{equation*}
    (y_p^{\prime\prime}+p\cdot y_p^\prime+q\cdot y_p)+(y_h^{\prime\prime}+p\cdot y_h^\prime+q\cdot y_h)=
    f(x)+0=
    f(x)
\end{equation*}
Alltså, för att lösa den icke-homogena differentialekvationen \underline{måste} man även lösa den homogena.
Den totala lösningen ges av summan $y_p+y_h$ där $y_p$ kallas för \underline{partikulärlösning} och $y_h$ kallas för \underline{homogenlösning}.
Hur hittar man homogenlösningen $y_h$, dvs. lösningen till differentialekvationen:
\begin{equation*}
    y^{\prime\prime}+p\cdot y^\prime+q\cdot y=0
\end{equation*}
för givna tal $p,q\in\mathbb{R}\, (\mathbb{C})$?
Testa ansatsen $y=e^{r\cdot x},\, r\in\mathbb{C}$.
\begin{equation*}
    \Rightarrow(e^{r\cdot x})^{\prime\prime}+p(e^{r\cdot x})^\prime+q(e^{r\cdot x})=
    r^2e^{rx}+pre^{rx}+qe^{rx}=
    (r^2+pr+q)\cdot e^{rx}=0
\end{equation*}
Men detta kan bara vara sant om $r^2+pr+q=0$ och det gäller att hitta det tal $r$ som löser den ekvationen.
Man brukar kalla $r^2+pr+q=0$ för den \underline{karakteristiska ekvationen} tillhörande den homogena differential ekvationen $y^{\prime\prime}+py^\prime+q=0$.
Den karakteristiska ekvationen har \underline{alltid} två lösningar räknade med multiplicitet (algebrans fundamentalsats) och vi får tre olika fall:
(givet att $p,q\in\mathbb{R}$)
\begin{enumerate}
    \item Två reella rötter $r_1$ och $r_2$.\\
          \begin{equation*}
              \Rightarrow y_h=A\cdot e^{r_1x}+B\cdot e^{r_2x},\, A,B\in\mathbb{R}
          \end{equation*}
    \item En dubbelrot $r(=r_1=r_2)$.\\
          \begin{equation*}
              \Rightarrow y_h=A\cdot e^{r}+Bxe^{rx},\, A,B\in\mathbb{R}
          \end{equation*}
    \item Två komplexkonjugerande rötter $r_{1,2}=k\pm i\omega$.
          \begin{equation*}
              \Rightarrow y_h=C\cdot e^{(k+i\omega)x}+D\cdot e^{(k-\omega i)x}=
              C\cdot e^{kx}\cdot e^{i\omega x}+De^{kx}\cdot e^{-i\omega x}=
          \end{equation*}
          \begin{equation*}
              \{e^{i\omega x}=\cos(\omega x)+i\sin(\omega x)\}=
              ...=
              e^{kx}(A\cos(\omega x)+B\sin(\omega x))
          \end{equation*}
\end{enumerate}
Att lösa den karakteristiska ekvationen och därefter identifiera vilket av fallen 1., 2. eller 3. man har ger ett "recept" som \underline{alltid} kan användas för att lösa homogena och linjära andra ordningens differentialekvationer med konstanta koefficienter.

\paragraph*{Ex} Lös differential ekvationen
$   \left\lbrace
    \begin{matrix}
        2y^{\prime\prime}+5y^\prime-3y=0 \\
        y(0)=0                           \\
        y^\prime(0)=1
    \end{matrix}
    \right.$
\subparagraph*{Lösning}
Differentialekvationen är homogen av andra ordningen med konstanta koefficienter och vi vill lösa den karakteristiska ekvationen.
\begin{equation*}
    2r^2+5r-3=0\Leftrightarrow
    r^2+\frac{5}{2}r-\frac{3}{2}=0\Leftrightarrow
    r^2+2\cdot\frac{2}{4}r+(\frac{5}{4})^2-(\frac{5}{4})^2-\frac{3}{2}=0\Leftrightarrow
\end{equation*}
\begin{equation*}
    (r+\frac{5}{4})^2=
    \frac{25}{16}+\frac{3}{2}=
    \frac{25}{16}+\frac{24}{16}=
    \frac{49}{16}\Leftrightarrow
    r_{1,2}=-\frac{5}{4}\pm\sqrt{\frac{49}{16}}=
    -\frac{5}{4}\pm\frac{7}{4}
\end{equation*}
så vi får två reella rötter $r_1=-3$ och $r_2=\frac{1}{2}$.
En allmän lösning ges därför av:
\begin{equation*}
    y_h=A\cdot e^{-3x}+Be^{\frac{x}{2}},\, A,B\in\mathbb{R}
\end{equation*}
Talen $A$ och $B$ måste respektera villkoren $y(0)=0$ och $y^\prime(0)=1$.
\begin{equation*}
    \left.
    \begin{matrix}
        y(0)=0\Rightarrow y_h(0)=Ae^{0}+Be^{0}=A+B=0 \\
        y^\prime(0)=1\Rightarrow y_h(0)=-3Ae^{0}+\frac{B}{2}e^0=\frac{B}{2}-3A=1
    \end{matrix}
    \right\rbrace
\end{equation*}
\begin{equation*}
    A=-B\Rightarrow\frac{B}{2}-3(-B)=1\Leftrightarrow
    \frac{B}{2}+3B=1\Leftrightarrow
    \frac{7}{2}B=1\Leftrightarrow
\end{equation*}
\begin{equation*}
    B=\frac{2}{7}\Rightarrow
    A=-\frac{2}{7}\Leftrightarrow
    B=\frac{2}{7}\Rightarrow A=-\frac{2}{7}
\end{equation*}
så vi får slutligen lösningen:
\begin{equation*}
    y=-\frac{2}{7}e^{-3x}+\frac{2}{7}e^\frac{x}{2}=\frac{2}{7}(e^\frac{x}{2}-e^{-3x})\, \Box
\end{equation*}
För en differentialekvation av typen $y^{\prime\prime}+py^\prime+qy=f(x)$, $p,q\in\mathbb{R}$ så är lösningen på formen $y=y_p+y_h$ och vi kan hitta $y_h$.
\\
\\
Hur hittar man partikulärlösningen?
Kan vara \underline{supersvårt}, men om inte högerledet $f(x)$ är för komplicerat kan man sätta upp bra ansatser och kontrollera att det funkar.
Låt $P_n(x)=p_0+p_1x+p_2x^2+...+p_nx^n$ (n:te-grads polynom), $A_n(x)=a_0+a_1x+...+a_nx^n$ och $B_n(x)=b_0+b_1x+...+b_nx^n$ där talen $a_0,...,a_n,b_0,...,b_n\in\mathbb{R}$ är okända.
\begin{enumerate}
    \item om $f(x)=P_n(x)$ ansätt $y_p=x^m A_n(x)$
    \item $f(x)=P_n(x)e^{rx}$ ansätt då $y_p(x)=x^mA_n(x)e^{rx}$
    \item $f(x)=P_n(x)e^{rx}\cos(kx)$ eller $f(x)=P_n(x)e^{rx}\sin(kx)$ ansätt $y_p(x)=x^m e^{rx}(A_n(x)\cos(kx)+B_n(x)\sin(kx))$
\end{enumerate}
Talet $m$ i faktorn $x^m$ väljs så liten som möjligt, $m=0,1,2$, för att se till att $y_p$ \underline{inte} överlappar med homogenlösningen $y_h$.
Kräver mycket träning!

\paragraph*{Ex (19.6 chapter review 26)}
Lös begynnelsevärdesproblemet $
    \left\lbrace
    \begin{matrix}
        2y^{\prime\prime}+5y^\prime-3y=6+7e^{\frac{x}{2}} \\
        y(0)=0                                            \\
        y(0)^\prime=1
    \end{matrix}
    \right.$
\subparagraph*{Lösning}
Vi har redan löst det homogena problemet (tidigare exempel) så gäller nu att hitta partikulärlösning.
Prova ansatsen $y_p(x)=A+Bxe^\frac{x}{2}$ vilken ger derivatorna.
\begin{equation*}
    y_p^\prime(x)=0+Be^\frac{x}{2}+Bx\frac{1}{2}e^\frac{x}{2}=B(1+\frac{x}{2})e^\frac{x}{2}
\end{equation*}
\begin{equation*}
    y^{\prime\prime}_p(x)=B\frac{1}{2}e^\frac{x}{2}+B(1+\frac{x}{2})\frac{1}{2}e^\frac{x}{2}=B(1+\frac{x}{4})e^\frac{x}{2}
\end{equation*}
Sätt in i differentialekvationen och försök identifiera vad talen $A$ och $B$ måste vara.
\begin{equation*}
    2y_p^{\prime\prime}+5y^\prime_p-3y_p=
    2b(1+\frac{x}{4})e^{\frac{x}{2}}+5B(1+\frac{x}{2})e^\frac{x}{2}-3(A+Bxe^\frac{x}{2})=
\end{equation*}
\begin{equation*}
    -3A+(2B+5B)e^\frac{x}{2}+(\frac{2}{4}B+\frac{5}{2}B-3B)x^\frac{x}{2}=
    -3A+7Be^\frac{x}{2}+(\frac{2+10-12}{4}B)x^\frac{x}{2}=
\end{equation*}
\begin{equation*}
    -3A+7Be^\frac{x}{2}=
    6+7e^\frac{x}{2}\Leftrightarrow
    A=-2,\, B=1
\end{equation*}
För homogenlösningen vet vi att $y_h(x)=C_1e^\frac{x}{2}+C_2e^{-3x}$ och den allmänna lösningen blir
\begin{equation*}
    y(x)=
    y_p(x)+y_h(x)=
    -2+xe^\frac{x}{2}+C_1e^\frac{x}{2}+C_2e^{-3x}
\end{equation*}
Vi vet också att $y^\prime(0)=1$:
\begin{equation*}
    \left.
    \begin{matrix}
        y(0)=0\Rightarrow-2+C_1+C_2=0 \\
        y^\prime(x)=1\Rightarrow 1+\frac{C_2}{2}-3C_2=1
    \end{matrix}
    \right\rbrace
\end{equation*}
$\Rightarrow...\Rightarrow C_1=\frac{12}{7},\, \frac{2}{7}$ och vi får
\begin{equation*}
    y(x)=-2+xe^\frac{x}{2}+\frac{12}{7}e^\frac{x}{2}+\frac{2}{7}e^{-3x}\, \Box
\end{equation*}
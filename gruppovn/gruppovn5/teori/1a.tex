För att visa att $v_1,v_2,v_3$ är linjärt oberoende så antar vi att $x_1v_1+x_2v_2+x_3v_3=0$.
Detta ger oss den följande matrisen:
\begin{equation*}
    \begin{pmatrix}
        1&2&0&0\\
        0&3&2&0\\
        2&1&1&0
    \end{pmatrix}
    \thicksim
    \begin{pmatrix}
        1&2&0&0\\
        0&-3&1&0\\
        0&3&2&0
    \end{pmatrix}
    \thicksim
    \begin{pmatrix}
        1&2&0&0\\
        0&1&0&0\\
        0&0&1&0
    \end{pmatrix}
    \thicksim
    \begin{pmatrix}
        1&0&0&0\\
        0&1&0&0\\
        0&0&1&0
    \end{pmatrix}
\end{equation*}
Den slutgiltiga matrisen visar att alla vektorerna är linjärt oberoende till varandra.
\\\\
Då $det(A)\neq 0$ så är våra vektorer baser:
\begin{equation*}
    det\begin{pmatrix}
        1&2&0\\
        0&3&2\\
        2&1&1
    \end{pmatrix}
    =1(3\cdot 1-2\cdot 1)-2(0\cdot 1-2\cdot 2)+0(0\cdot 1-2\cdot 2)=9\neq 0
\end{equation*}
Då våra vektorer är baser innebär det att man kan skriva alla 3-vektorer som en linjärkombination av våra vektorer.
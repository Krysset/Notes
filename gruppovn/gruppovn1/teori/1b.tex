Med tre godtyckliga punkter på en linje, $P_{a}$, $P_{b}$ och $P_{c}$, kan vi bilda tre st vektorer, $\bm{v}$, $\bm{u}$ och $\bm{w}$.
Då punkterna ligger på en linje får alla vektorer samma riktning.
Alltså är minsta vinkeln, $\alpha$, blir då 0.
Uttrycket i roten ur blir då: 
\begin{center}
    $
    ||\bm{v}||^{2}-(\frac{|\bm{u}\cdot \bm{v}|}{||\bm{u}||})^{2} = 
    ||\bm{v}||^{2}-(\frac{||\bm{u}||\cdot||\bm{v}||cos(0)}{||\bm{u}||})^{2}=
    ||\bm{v}||^{2}-||\bm{v}||^{2}=0
    $
\end{center}
Detta leder till att vi får: $\frac{\sqrt{0}\cdot ||\bm{u}||}{2}=\frac{0}{2}=0$.
Alltså är arean av den triangeln som $P_{a}$, $P_{b}$ och $P_{c}$ bildar 0.
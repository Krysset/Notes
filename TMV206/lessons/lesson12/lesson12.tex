\paragraph{Proposition 6.17} Om $A$ är en $n\times n$-matris då är $det(A^{t})=det(A)$.

\paragraph{Proposition 6.19} $det(A)\neq 0 \Leftrightarrow A$ inverterbar.
Detta följer av vad vi har gjort för ekvationssystem.

\paragraph{Sats 6.22} Om $A,B$ är $n\times n$-matriser då är $det(AB)=det(A)\cdot det(B)$

\paragraph{Proposition 6.23} $det(A^{-1})=\frac{1}{det(A)}$
\subparagraph{Bevis}
$1=det(I_{n})=det(A\cdot A^{-1})=det(A)\cdot det(A^{-1}) \text{ } \blacksquare$
\clearpage
Med hjälp av permutationer kan vi beräkna $n\times n$-determinanter som en summa av $(n-1)\times (n-1)$-determinanter.
Givet en $n\times n$-matris $A$ betecknar vi med $A_{ij}$ $(n-1)\times (n-1)$-matrisen som fås av $A$ genom att ta bort rad $i$ och kolumn $j$.
Determinanten av $A$ kan beräknas genom att expandera längs en rad eller längs  en kolumn:
\begin{enumerate}
    \item[Längs en rad:] ~\\
        $det(A)=(-1)^{k+1}a_{k,1} det(A_{k,1})+(-1)^{k+2}a_{k,2}det(A_{k,2})+\ldots+(-1)^{k+n}det(A_{k,n})$
    \item[Längs en kol:] ~\\
        $det(A)=(-1)^{1+k}a_{1,k}det(A_{1,k})+(-1)^{2+k}a_{2,k}det(A_{2,k})+\ldots+(-1)^{n+k}a_{n,k}det(A_{n,k})$
\end{enumerate}

\chapter{Baser}
Vi har kallat $\bm{e}_{1},\bm{e}_{2},\ldots,\bm{e}_{n}$ för \underline{standardbasen för $\mathbb{R}^{n}$}.
Vi ska nu se vad en bas är.

\paragraph{Definition} Vektorerna $\bm{v}_{1},\bm{v}_{2},\ldots,\bm{v}_{m}$ i $\mathbb{R}^{n}$ sägs vara \underline{linjärt oberoende} om
\begin{equation*}
    c_{1}\bm{v}_{1}+c_{2}\bm{v}_{2}+\ldots+c_{m}\bm{v}_{m}\Rightarrow c_{1}=c_{2}=\ldots=c_{m}=0
\end{equation*}

\paragraph{Ex} Låt $\bm{v}_{1}=\bm{v}$ någon vektor och $\bm{v}_{2}=\bm{0}$.
Då är $0\cdot\bm{v}_{1}+2\cdot \bm{v}_{2}=\bm{0}$ så är $\bm{v}_{1},\bm{v}_{2}$ inte linjärt oberoende utom \underline{linjärt beroende}.

\paragraph{Proposition 7.4} $\bm{v}_{1},\bm{v}_{2}\neq \bm{0}$ är linjärt oberoende $\Leftrightarrow \bm{v}_{1},\bm{v}_{2}$ parallella.
\subparagraph{Bevis} Om $\bm{v}_{1},\bm{v}_{2}$ är parallella då finns $c\neq 0$ så att $\bm{v}_{1}=\bm{v}_{2}$.
Alltså är $1\cdot \bm{v}_{1}-c\cdot \bm{v}_{2} = \bm{0}$ så $\bm{v}_{1},\bm{v}_{2}$ är linjärt beroende.
Å andra sidan, om $\bm{v}_{1},\bm{v}_{2}$ är linjärt beroende så finns $c_{1},c_{2} \neq 0$ så att $c_{1}\bm{v}+c_{2}\bm{v}_{2}=\bm{0}$
Vi kan anta att $c_{1}\neq 0$ och alltså är $\bm{v}_{1}=-\frac{c_{2}}{c_{2}}\bm{v}_{2}$ vilket ger att $\bm{v}_{1},\bm{2}$ är parallella. $\blacksquare$

\paragraph{Ex} Ta reda på om $\bm{v}_{1}=\begin{pmatrix}1\\1\\0\end{pmatrix},\bm{v}_{2}=\begin{pmatrix}3\\2\\1\end{pmatrix}$ och $\bm{v}_{3}=\begin{pmatrix}4\\3\\1\end{pmatrix}$ är linjärt oberoende.
\subparagraph{Lösning} Antag att $x_{1}\bm{v}_{1}+x_{2}\bm{v}_{2}+x_{3}\bm{v}_{3}=\bm{0}$.
Måste $x_{1}=x_{2}=x_{3}=0$? 
Vi vill lösa detta ekvationssystem.
Totalmatrisen ges av:
\begin{equation*}
    \begin{pmatrix}
        1&3&4&0\\
        1&2&3&0\\
        0&1&1&0
    \end{pmatrix}
    \thicksim
    \begin{pmatrix}
        1&3&4&0\\
        0&-1&-1&0\\
        0&1&1&0
    \end{pmatrix}
    \thicksim
    \begin{pmatrix}
        1&3&4&0\\
        0&1&1&0\\
        0&0&0&0
    \end{pmatrix}
\end{equation*}
Det finns oändligt många lösningar.
Speciellt finns det lösningar $x_{1},x_{2},x_{3}$ inte alla är noll.
Det betyder att $\bm{v}_{1},\bm{v}_{2},\bm{v}_{3}$ är linjärt beroende.\\
$\bm{v}_{1},\bm{v}_{2},\bm{v}_{3}$ linjärt beroende ty $\bm{v}_{1}+\bm{v}_{2}=\bm{v}_{3} \Leftrightarrow 1\cdot\bm{v}_{1}+1\cdot\bm{v}_{2}+(-1)\bm{v}_{3}=\bm{0}$

\paragraph{Proposition 7.4} Vektorerna $\bm{v}_{1},\bm{v}_{2},\ldots,\bm{v}_{m}$ är linjärt beroende $\Leftrightarrow$ En av vektorerna $\bm{v}_{1},\bm{v}_{2},\ldots,\bm{v}_{m}$ går att skriva som en linjärkombination av de andra.

\paragraph{Proposition 7.6} Låt $\bm{v}_{1},\bm{v}_{2},\ldots,\bm{v}_{m}$ vara vektorer i $\mathbb{R}^{n}$.
Vektorerna $\bm{v}_{1},\bm{v}_{2},\ldots,\bm{v}_{m}$ är linjärt oberoende $\Leftrightarrow$ ekvationssystem $\begin{pmatrix}\bm{v}_{1}&\bm{v}_{2}&\ldots&\bm{v}_{m}\end{pmatrix}\bm{x}=\bm{0}$ har bara lösningen $\bm{x}=\bm{0}$.

\paragraph{Ex} Visa att $\bm{v}_{1}=\begin{pmatrix}1\\2\\3\end{pmatrix}, \bm{v}_{2}=\begin{pmatrix}4\\5\\6\end{pmatrix}, \bm{v}_{3}=\begin{pmatrix}7\\8\\3\end{pmatrix}$ är linjärt oberoende.
\subparagraph{Lösning} 
\begin{equation*}
    det\begin{pmatrix}
        1&4&7\\
        2&5&8\\
        3&8&3
    \end{pmatrix}
    =
    1\cdot\begin{vmatrix}5&8\\8&3\end{vmatrix}
    -4\cdot\begin{vmatrix}2&8\\3&3\end{vmatrix}
    +7\begin{vmatrix}2&5\\3&8\end{vmatrix}
    =(15-64)-4(6-24)+7(16-15)=34\neq 0
\end{equation*}
$det(A)\neq 0 \Leftrightarrow$ Varje ekvationssystem $A\bm{x}=\bm{b}$ har en unik lösning ($\bm{x}=A^{-1}\bm{b}$).
Speciellt har $A\bm{x}=\bm{0}$ en unik lösning vilket är $\bm{x}=\bm{0}$.
Om vi använder detta på $A=\begin{pmatrix}
    \bm{v}_{1}&\bm{v}_{2}&\bm{v}_{3}
\end{pmatrix}$
så får vi att $\bm{v}_{1},\bm{v}_{2},\bm{v}_{3}$ är linjärt oberoende.

\paragraph{Definition} Låt $\bm{v}_{1},\bm{v}_{2},\ldots,\bm{v}_{m}$  vara vektorer o $\mathbb{R}^{n}$. 
$(\bm{v}_{1},\bm{v}_{2},\ldots,\bm{v}_{m})$ sägs vara en \underline{bas för $\mathbb{R}^{n}$} om varje vektor $\bm{v}\in\mathbb{R}^{n}$ går att på ett unikt sätt skriva som en linjärkombination av $\bm{v}_{1},\bm{v}_{2},\ldots,\bm{v}_{m}$:
\begin{equation*}
    \bm{v}=a_{1}\bm{v}_{1}+a_{2}\bm{v}_{2}+\ldots+a_{m}\bm{v}_{m}
\end{equation*}
$a_{1},a_{2},\ldots,a_{m}$ kallas för $\bm{v}$'s \underline{koordinater i basen}.

\paragraph{Proposition 7.11} $\bm{v}_{1},\bm{v}_{2},\ldots,\bm{v}_{m}$ är en bas $\Leftrightarrow \bm{v}_{1},\bm{v}_{2},\ldots,\bm{v}_{m}$ linjärt oberoende\\
Varje vektor går att skriva som en linjärkombination av $\bm{v}_{1},\bm{v}_{2},\ldots,\bm{v}_{m}$

\paragraph{Ex} Standardbasen $\bm{e}_{1},\bm{e}_{2},\ldots,\bm{e}_{n}$ är en bas för $\mathbb{R}^{n}$.
\subparagraph{Bevis} 
\begin{enumerate}
    \item $\bm{e}_{1},\bm{e}_{2},\ldots,\bm{e}_{n}$ är linjärt oberoende ty ekvationssytemet $\begin{pmatrix}
        \bm{e}_{1}&\bm{e}_{2}&\ldots&\bm{e}_{n}
    \end{pmatrix}\bm{x}=\bm{0}$ har bara lösningen $\bm{x}=\bm{0}$.
    \item En allmän vektor $\bm{v}=\begin{pmatrix}v_{1}\\v_{2}\\v_{3}\end{pmatrix}$ 
    går att skriva som en linjärkombination av 
    $\bm{e}_{1},\bm{e}_{2},\ldots,\bm{e}_{n}$ alltså $\bm{v}=v_{1}\bm{e}_{1}+v_{2}\bm{e}_{2}+\ldots+v_{n}\bm{e}_{n}$
\end{enumerate}
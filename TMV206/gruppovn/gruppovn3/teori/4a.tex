\begin{enumerate}[label=Fall \arabic*:]
    \item  Lösningsmängd med en parameter\\
        Om vi antar att följande stämmer 
        \begin{equation*}
            A_{1} \neq kA_{2}\text{, }B_{1} \neq kB_{2}\text{, }C_{1} \neq kC_{2} \text{ och } D_{1} \neq kD_{2}\text{, }k\in\mathbb{R}
        \end{equation*}
        så innebär det att alla ekvationer i systemet skall ge en lösning på en utav de okända variablerna.
        Då två av tre variabel är kända innebär det att vi får en lösning med parameter, alltså är lösningsmängden en linje.
        \\
    \item  Lösningsmängd med två parametrar\\
        Om ekvation 1 och 2 är lika, dvs 
        \begin{equation*}
            A_{1} = kA_{2}\text{, }B_{1} = kB_{2}\text{, }C_{1} = kC_{2} \text{ och } D_{1} = kD_{2}\text{, }k\in\mathbb{R}
        \end{equation*}
        så finns det inte tillräckligt med information för att lösa ut två variabler.
        Och därför får man alltså en variabel i form utav två parametrar, vilket resulterar i ett plan. (Plan på parameterform.)
        \\
    \item Saknar lösningsmängd\\
        Ifall ekvation 2 är en multipel av ekvation 1 med ett annat värde på $D_{2}$, alltså
        \begin{equation*}
            A_{1} = kA_{2}\text{, }B_{1} = kB_{2}\text{, }C_{1} = kC_{2} \text{ och } D_{1} \neq kD_{2}\text{, }k\in\mathbb{R}
        \end{equation*}
        så kommer ekvationssystemet att sakna lösningar och därmed inte ha någon lösningsmängd.
\end{enumerate}
\begin{enumerate}[label=Fall \arabic*:]
    \item Lösningsmängd utan parameter\\
        För en lösningsmängd utan parameter måste 
        \begin{equation*}
            A_{1} \neq kA_{2}\text{, }B_{1} \neq kB_{2}\text{ och }C_{1} \neq kC_{2} \text{, }k\in\mathbb{R}
        \end{equation*}
        \begin{equation*}
            A_{1} \neq mA_{3}\text{, }B_{1} \neq mB_{3}\text{ och }C_{1} \neq mC_{3} \text{, }m\in\mathbb{R}
        \end{equation*}
        \begin{equation*}
            A_{2} \neq nA_{3}\text{, }B_{2} \neq nB_{3}\text{ och }C_{2} \neq nC_{3} \text{, }n\in\mathbb{R}
        \end{equation*}
        Varje okänd variabel kommer få en lösning vilket ger oss en lösningsmängd av en punkt (i $\mathbb{R}^{3}$).
    \item Lösningsmängd med en parameter\\
        Det blir en parameters lösning då en av ekvationerna är en ekvation som består av multiplar av de andra två ekvationerna.
        Alltså $A_{3}=cA_{1}+kA_{2}$ (inte nödvändigt i denna ordning) för varje $A,B,C,D$. Lösningsmängden bildar en linje.
        
    \item Lösningsmängd med två parametrar\\
        För en lösning med två parametrar måste alla ekvationer som ingår i ekvationssystemet vara multipler av samma ekvation dvs
        \begin{equation*}
            A_{1} = kA_{2} = cA_{3}\text{, }B_{1} = kB_{2} = cB_{3}\text{, }C_{1} = kC_{2} = cC_{3}\text{, }D_{1} = kD_{2} = cD_{3} \text{,} k,c\in\mathbb{R}
        \end{equation*}
        då detta kan representeras som att vara 3 "överlapande" plan (dvs samma plan). Då blir det en lösningsmängd av två parametrar, alltså planet.

    \item Saknar lösningsmängd\\
        Om 2 eller 3 av ekvationerna är multipler av varandra, 
        men $D_{1} \neq kD_{2} \neq cD_{3}$ (där $k,c\in\mathbb{R}$) kommer det inte finnas några lösningar,
        då det blir 2 eller 3 plan som är parallela och aldrig korsas, och då finns inga punkter som uppfyller systemet.
\end{enumerate}